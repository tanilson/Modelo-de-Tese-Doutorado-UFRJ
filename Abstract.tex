\begin{abstract}
Golumbic, Lipshteyn e Stern definiram em 2009 a classe de grafos EPG, uma classe de grafos de intersecção baseada na intersecção de arestas em caminhos sobre uma grade. Um grafo EPG $G$ é um grafo que admite uma representação onde seus vértices correspondem a caminhos em uma grade $Q$, tal que dois vértices de $G$ são adjacentes se e somente se seus caminhos correspondentes em $Q$ tem pelo menos uma aresta comum. Se os caminhos na representação tem no máximo $k$ mudanças de direção (dobras), dizemos que  essa é uma representação $B_k$-EPG. Uma coleção $C$ de conjuntos satisfaz a propriedade Helly quando toda subcoleção de $C$ que é mutuamente intersectante possui no mínimo um elemento comum. Neste trabalho mostramos que as classes $B_k$-EPG e $B_k$-EPG-Helly nem sempre coincidem, demonstramos que todo grafo possui uma representação $B_k$-EPG-Helly e estudamos o problema de reconhecimento de grafos $B_1$-EPG cujo conjunto de aresta-intersecção de caminhos na grade satisfaz a propriedade Helly. Além disso, também são apresentadas algumas classes de grafos para os quais os resultados podem se estender. 


Palavras-chave: Aresta-intersecção de caminhos sobre uma grade, Propriedade Helly, Grafos de Intersecção, $NP$-completude, Dobra simples.
\end{abstract}