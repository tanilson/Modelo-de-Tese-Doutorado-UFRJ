\chapter{Considerações Finais}\label{conclusao}

\begin{flushright}
\begin{minipage}[t][0cm][b]{0.47\textwidth}
\emph{
Se eu vi mais longe, foi por estar sobre ombros de gigantes.}
\end{minipage}

\rule[0cm]{7cm}{0.03cm}%{largura}{espessura}

Isaac Newton
\end{flushright}

\section{Discussão, Resultados Iniciais e Perspectivas Futuras}

Definir uma perspectiva de estudos para grafos EPG e para grafos VPG;

 O problema de reconhecer se um grafo possui uma representação $B_{k}-$EPG é um problema em aberto para $k\geq 3$, i.e. dado um grafo $G$, qual é o menor $k$ tal que $G$ possui uma representação $B_{k}-$EPG?

O tamanho mínimo da grade para representar diferentes classes de grafos EPG também é problema em aberto.

Abordagens utilizadas

problemas em aberto

perspectivas de trabalhos futuros

Breve descrição do trabalho
Um resumo das conclusões da pesquisa
Relação com as evidências existentes
Impactos e implicações que os resultados podem gerar
Limitações do estudo
Trabalhos futuros

Artigos publicados:

BORNSTEIN, C. F. ; SANTOS, T. D. ; SOUZA, U. S. ; SZWARCFITER, J. L. . A Complexidade do Reconhecimento de Grafos B1-EPG-Helly. In: 50º SBPO - Simpósio Brasileiro de Pesquisa Operacional, 2018, Rio de Janeiro. Cidades Inteligentes: Planejamento Urbano, Fontes Renováveis e Distribuição de Recursos, 2018.

BORNSTEIN, C. F. ; SANTOS, T. D. ; SOUZA, U. S. ; SZWARCFITER, J. L. Sobre a Dificuldade de Reconhecimento de Grafos B1-EPG-Helly. In: XXXVIII Congresso da Sociedade Brasileira de Computação, 2018, Natal - RN. Computação e Sustentabilidade, 2018. p. 113-116.

BORNSTEIN, C. F. ; SANTOS, T. D. ; SOUZA, U. S. ; SZWARCFITER, J. L. The complexity of B1-EPG-Helly graph recognition. In: VIII Latin American Workshop On Cliques in Graphs (LAWCG), ICM 2018 Satellite Event, 2018, Rio de Janeiro. Program and Abstracts, 2018. p. 69.

BORNSTEIN, C. F. ; GOLUMBIC, M.C. ; SANTOS, T. D. ; SOUZA, U. S. ; SZWARCFITER, J. L.  The complexity of B1-EPG-Helly graph recognition. In:  X Latin and American Algorithms, Graphs and Optimization Symposium (LAGOS),  2019, Belo Horizonte - MG. (Submissão).