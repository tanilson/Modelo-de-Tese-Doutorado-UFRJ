\chapter{Introdução}

\begin{flushright}
\begin{minipage}[t][0cm][b]{0.47\textwidth}
\emph{Se não podes entender, crê para que entendas. A fé precede, o intelecto segue.}
\end{minipage}

\rule[0cm]{7cm}{0.03cm}%{largura}{espessura}

Santo Agostinho
\end{flushright}


\section{Motivação e objetivos}

Um grafo EPG $G$ é um grafo que admite uma representação em que seus vértices são representados por caminhos de uma grade $Q$, tal que dois vértices de $G$ são adjacentes se e somente se os caminhos correspondestes tem no mínimo uma aresta em comum.

%Apesar dos problemas de Teoria de Grafos serem algumas vezes puramente matemáticos, é possível encontrar alguns problemas com motivação prática. 
O estudo de grafos EPG tem motivação relacionada com o problema de \textit{design} VLSI que combina a noção de grafos de intersecção de arestas de caminhos em uma árvore com um modelo de \textit{layout} de grade VLSI~\cite{golumbic2009}. O número de dobras em um circuito integrado pode aumentar a área de \textit{layout} e consequentemente aumentar o custo de produção do microchip. 
Essa é uma das principais aplicações que instigam a  pesquisa sobre representações EPG de algumas famílias de grafos quando existem restrições no número de dobras nos caminhos usados na representação.
Outras aplicações e detalhes sobre problemas de \textit{layout} de circuitos podem ser encontrados em~\cite{bandy1990, molitor1991}.   

Um grafo é $ B_k$-EPG se ele admite uma representação (sobre uma grade) em que cada caminho possua no máximo $k$ dobras. A título de exemplo a Figura~\ref{fig:trianguloepgRepresentacao}(a) retrata um grafo $C_3$, a Figura~\ref{fig:trianguloepgRepresentacao}(b) retrata uma representação EPG do grafo $C_3$ onde os caminhos não possuem dobra e a  Figura~\ref{fig:trianguloepgRepresentacao}(c) retrata uma representação com 1 dobra do grafo $C_3$. Consequentemente, $C_3$ é um grafo  $B_0$-EPG. %De forma mais geral, grafos $B_0$-EPG coincidem com grafos de intervalo~\cite{golumbic2009}.

O \emph{número de dobras} de uma classe de grafos é o menor  $k$ para o qual todos os grafos na classe possuem uma representação $B_k$-EPG. Grafos de intervalo possuem número de dobras $0$~\cite{golumbic2009}, árvores possuem  número de dobras $1$~\cite{golumbic2009} e  grafos outerplanar possuem número de dobras $2$~\cite{daniel2014b}. O número de dobras da classe dos grafos planares é um problema em aberto, porém sabe-se que ele é $ 3 $ ou $4$~\cite{daniel2014b}. %Apesar de existirem demonstrações de que o número de dobras não é maior que 4 para a classe dos grafos planares, também não se conhece algum grafo planar que não possa ser representado com 3 dobras.

A classe dos grafos EPG tem sido estudada em diversos trabalhos, tais como  \cite{alcon2016, Asinowski2009, cohen2014, golumbic2009, heldt2014,  martin2017}, entre outros. As investigações frequentemente abordam caracterizações com relação número de dobras das representações de um grafo. A respeito da complexidade de reconhecimento de grafos $B_k$-EPG, somente a complexidade de reconhecimento de três subclasses de grafos EPG foi determinada:
 grafos $B_0$-EPG podem ser reconhecidos em tempo polinomial, uma vez que correspondem a classe de grafos de intervalo, ver~\cite{booth1976, golumbic2009}. Em contraste, o reconhecimento das classes $B_1$-EPG e $B_2$-EPG é $NP$-completo, ver~\cite{heldt2014, martin2017}.

\input{./includes/include-img/representacoesepgtriangulo.tex}


%A  collection $C$ of sets satisfies the Helly property when every sub-collection of $ C $ that is pairwise intersecting has at least one common element. The Helly property has this name in honor of the great Austrian mathematician Eduard Helly, who in 1923 proposed his famous theorem concerning the relation of intersecting sets.

%The study of the Helly property is useful in very diverse areas of science, and we can enumerate applications in semantics, code theory, computational biology, database, image processing, graph theory, optimization, and linear programming \cite{dourado2009}.

%Note that the representation of Figure~\ref{fig:trianguloepgRepresentacao}(b) satisfies the Helly property, while the representations of Figures~\ref{fig:trianguloepgRepresentacao}(c) and~\ref{fig:trianguloepgRepresentacao}(d) do not satisfy it.

Este trabalho propõe o estudo de grafos que possuem uma representação EPG-Helly. 
A propriedade Helly relacionada com representações EPG foi  estudada em~\cite{golumbic2009} e \cite{golumbic2013}. Em particular, esses trabalhos determinaram um parâmetro conhecido como número de Helly forte  para grafos $B_1$-EPG. 

Estão no escopo de interesse deste trabalho os seguintes tópicos:

\begin{itemize}
    
    \item Determinar a complexidade de reconhecimento de grafos $B_1$-EPG-Helly;
    \item Determinar limites superiores e/ou inferiores para os parâmetros número de Helly e número de Helly forte em grafos EPG e EPG-Helly;
    
    \item Estudar os parâmetros número de Helly e número de Helly forte também em grafos de intersecção de vértices em caminhos sobre grade (VPG e VPG-Helly);
    
    \item Encontrar classes de grafos para os quais os resultados possam se estender.
\end{itemize}



% \section{Motivação}
% Por que pesquisar?

% Qual a relevância do problema e por que escolher esse tema?

% Qual é o problema estudado?
% Onde ele acontece?
% Quem observou ou observa sua ocorrência?
% Por que isso é importante e deve ser solucionado?

% \section{Objetivo}

% Quais as finalidades intelectuais?

% verificar... compreender... analizar...
% comparar...

\section{Organização do texto}

No Capítulo 2 apresentaremos algumas definições básicas sobre grafos juntamente com uma breve explicação sobre a propriedade Helly. Além disso, o capítulo aborda uma breve discussão sobre problemas de caminhos em grade.

O Capítulo 3 será dedicado à definição do problema estudado, análise de algumas representações EPG básicas e demonstração da $NP$-completude do problema de reconhecimento de grafos $B_1$-EPG-Helly. São publicações resultantes desta pesquisa, os seguintes escritos:

\begin{enumerate}
    \item BORNSTEIN, C. F.; SANTOS, T. D.; SOUZA, U. S.; SZWARCFITER, J. L. A Complexidade do Reconhecimento de Grafos B1-EPG-Helly. In: 50º SBPO - Simpósio Brasileiro de Pesquisa Operacional, 2018, Rio de Janeiro. Cidades Inteligentes: Planejamento Urbano, Fontes Renováveis e Distribuição de Recursos, 2018.

     \item BORNSTEIN, C. F.; SANTOS, T. D.; SOUZA, U. S.; SZWARCFITER, J. L. Sobre a Dificuldade de Reconhecimento de Grafos B1-EPG-Helly. In: XXXVIII Congresso da Sociedade Brasileira de Computação, 2018, Natal - RN. Computação e Sustentabilidade, 2018. p. 113-116.

     
     \item BORNSTEIN, C. F.; SANTOS, T. D.; SOUZA, U. S.; SZWARCFITER, J. L. The complexity of B1-EPG-Helly graph recognition. In: VIII Latin American Workshop On Cliques in Graphs (LAWCG), ICM 2018 Satellite Event, 2018, Rio de Janeiro. Program and Abstracts, 2018. p. 69.

     
     \item BORNSTEIN, C. F.; GOLUMBIC, M.C.; SANTOS, T. D.; SOUZA, U. S.; SZWARCFITER, J. L.  The complexity of B1-EPG-Helly graph recognition. %In:  45th International Workshop on Graph-Theoretic Concepts in Computer Science,  2019, Vall de Núria, Catalonia, Spain. 
     (Submited).
     
\end{enumerate}


Por fim, no Capítulo 4, discutimos os principais resultados obtidos para grafos EPG. Adicionalmente, teremos
também as considerações finais sobre o trabalho aqui apresentado com algumas perspectivas e ideias sobre o problema e possíveis direcionamentos para novos estudos de trabalhos futuros.